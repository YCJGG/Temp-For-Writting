



# p1
{W}{ith} the prevalence of touchscreens in consumer electronics (ranging from portable devices to large home appliance),
 human-machine interactions in free-hand drawings are becoming evermore facilitated and promoted. 
 The input of sketches is a succinct, convenient and efficient means of for visually recording ideas, and can beat hundreds of words in some scenarios. 
 Previous studies range from sketch recognition
  \cite{sketchanet}, sketch-based image retrieval(SBIR) \cite{SaavedraB10,2012sketchhash,Yu2016shoes,SangkloyBHH16,SongYSXH17,EitzHBA10},
  to sketch-image generation. 

# p2 : 总述sketch research 存在的问题

Despite great progress made, problems concerning sketches are very challenging due to the facts that:(i)Free-hand sketches are inherently abstract and iconic.
For SBIR and sketch-to-image generation, sketch domain and natural image domain are two distinctive domains with heterogeneous feature representations and distributions. The former is characterised by 
sparse black line drawing with white background while the latter consists of dense color pixels. Hence, it is difficult to explore their semantic 
relevance in sufficient detail to bridge the domain gap. For sketch recognition, the highly abstract sketches consist of lines instead of colored pixels, making them lack visual cues.
(ii) Sketches often display a varied levels of abstraction, sophistication and deformation, which is because same objects (maybe without any reference photo) can be drawn in total different styles. 
(iii) Sketch researches lack freely available sketch data. Compared with natural image based research whose million-scale datasets had been accessible for almost a decade, most existing 
sketch-based research can only utilize sub-million level crowd-sourced sketch datasets.

# p3 :简述前人的工作和存在的问题

#p4 : 简述本文的工作和解决的问题